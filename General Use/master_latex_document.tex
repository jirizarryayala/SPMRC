\documentclass{article}

%preamble here for options

%-----------------------------------------------------------------------------%
% DEFINITIONS:
%
% include \usepackage here
%-----------------------------------------------------------------------------%
\usepackage{authblk}
\usepackage{amsmath}
\usepackage{amssymb}
\usepackage{amsthm}
\usepackage{array}
\usepackage{epsfig}
\usepackage{graphicx}
\usepackage{xy}
%\input{bob_macros.tex}


%\usepackage{draftwatermark}
%\SetWatermarkText{University of Washington}
%\SetWatermarkScale{1}
%\SetWatermarkAngle{0}
%%\SetWatermarkColor[rgb]{0.7,0,0.7}
%\SetWatermarkLightness{.8}
%-----------------------------------------------------------------------------%
% HYPERREF: plain black hypertext references for ref's and cite's.
%-----------------------------------------------------------------------------%
\usepackage[pdftex,  linkcolor=blue, citecolor=black, filecolor=black, urlcolor=blue,colorlinks] {hyperref} % letterpaper, pdfusetitle, plainpages=false,bookmarks, bookmarksnumbered, 
\usepackage{relsize}   %for Hmisc tables
\usepackage{multirow}   % for multiple rows in tabular
\usepackage{rotating}           % for sideways tables/figures
\usepackage{pdflscape}        % for landscaping for large tables and figures
\usepackage{ctable}
\usepackage{longtable}  %to create large tables 
%\usepackage[T1]{fontenc}
%\usepackage[scaled]{uarial}    % aerial
\usepackage[scaled]{helvet}     % helvetica
%\renewcommand*\familydefault{\sfdefault} %% Only if the base font of the document is to be sans serif
\usepackage{subfig} % for using sub-plot
\usepackage{wrapfig} % to wrap figures
\usepackage{float}
% tikz & pgf
%\usepackage{tikz}
%\usepackage{pgfkeys}
%\usetikzlibrary[arrows,decorations.pathmorphing,backgrounds,positioning,fit,petri]
%\usetikzlibrary{mindmap} %
%\usetikzlibrary{er}
%\usetikzlibrary{matrix} % LATEX and plain TEX
%\usetikzlibrary{shapes.multipart}
\usepackage{setspace}
%%%%%%%%%%%%%%%%%%%%%%%%%%%%%%%%%%%%%%%%%%%%%%%%%%%%%%%%%%%%%%%%%%%%%%%%%%%%%%%%
\usepackage{fancyhdr} %Page style

\pagestyle{fancy}
\lhead{Tulane University - BIMI-6100\\ \today}
%\chead{}
\rhead{}
%\lfoot{}
%\cfoot{}
%\rfoot{}
\marginparwidth = 35pt %35pt
\textwidth = 440pt %390pt
\headwidth = 440pt
% one inch + \hoffset
% one inch + \voffset
\oddsidemargin = 15pt %31pt
% \topmargin = 20pt
% \headheight = 12pt
% \headsep = 25pt
% \textheight = 592pt
% \textwidth = 390pt
% \marginparsep = 10pt
% \marginparwidth = 35pt
% \footskip = 30pt
% \marginparpush = 7pt (not shown)
% \hoffset = 0pt
% \voffset = 0pt
% \paperwidth = 597pt
% \paperheight = 845pt
% http://en.wikibooks.org/wiki/LaTeX/Page_Layout

%Instructions: Use 8,000 characters or less including spaces (approximately two single-spaced, typed pages) to briefly describe your previous research training experience, your short-term academic and research objectives, your long-term career objectives and your plan to achieve these objectives. Please include your name and the date in the header of the document.

%Note: Please know that you can upload a new file to replace any previously uploaded file.

%This form will not lock until you submit the complete application.

%end of preamble, beginning of printable document

\title{BIMI-6100 Elements in Biomedical Informatics\\ Fall 2023}
\author{David Crosslin\\
\small  Tulane University School of Medicine \\
\small John W. Deming Department of Medicine \\
\small Division of Biomedical Informatics and Genomics
\small  New Orleans, LA}


\date{\today}
%-------------------------------------------------------------------------------
\begin{document}
\maketitle
\tableofcontents 
\listoffigures
\listoftables
\clearpage

%%------------------------------------------------------------------------------- 
\section{Hmisc \texttt{describe} function - rhc data}
\newpage
\input{subfiles/rhc.tex}
%%------------------------------------------------------------------------------- 
% read in input from R
\begin{landscape}
\thispagestyle{empty}
\section{Hmisc \texttt{summary} function - rhc data}
\input{subfiles/rhc_by_sex.tex}
\end{landscape}
%%------------------------------------------------------------------------------- 
%%------------------------------------------------------------------------------- 
\section{ols function output} 
\newpage
\input{subfiles/rhc_ols_summary.tex}
\input{subfiles/rhc_ols_anova.tex}

\begin{figure}[h]
    \centering
   \fbox{\includegraphics[scale=.50]{subfiles/rhc_ols_summary.png}}
    \caption{plot(summary(m))}
    \label{fig:plot(summary(m))}
\end{figure}

\begin{figure}[h]
    \centering
   \fbox{\includegraphics[scale=.50]{subfiles/rhc_ols_anova.png}}
    \caption{plot(anova(m))}
    \label{fig:plot(anova(m))}
\end{figure}


%\input{/google_drive/bime/classes/mebi_537/R/latex_output/m_summary.tex}
%\newpage

%\section{lrm fuction output}
%\input{subfiles/rhc_lrm.tex}

%\newpage
%\input{/google_drive/bime/classes/mebi_537/R/latex_output/m2.tex}
%\input{/google_drive/bime/classes/mebi_537/R/latex_output/m2_anova.tex}
%\input{/google_drive/bime/classes/mebi_537/R/latex_output/m2_summary.tex}


%\nocite{*}  %causes all items in the data base to be included in the references
\bibliographystyle{plain}                                                                                         
%put the name of your bibliography file here.
%\bibliography{refs}
\bibliography{/Users/davidcrosslin/support_files/bibliography/lit_crosslin.bib}
%\include{biography}
%-----------------------------------------------------------------------------%
\end{document}